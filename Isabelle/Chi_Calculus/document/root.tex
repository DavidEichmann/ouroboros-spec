\documentclass[a4paper,11pt]{article}

\usepackage{typearea}

\usepackage{isabelle,isabellesym}

\usepackage{latexsym}

\usepackage{pdfsetup}

\urlstyle{rm}
\isabellestyle{it}

\begin{document}

\title{The $\chi$-Calculus}
\author{Wolfgang Jeltsch}

\maketitle

\tableofcontents

\parindent 0pt\parskip 0.5ex

\section{Introduction}

The $\chi$-calculus is a process calculus with the following features:
\begin{itemize}

\item

Communication is always asynchronous. This is ensured by disallowing sequential composition of an
output operation and some follow-up process.

\item

There are two kinds of communication: unicast and broadcast. Unicast communication takes place over
channels, of which they are arbitrary many to choose from. All broadcast communication, on the other
hand, takes place over a single, separate medium, which we call the ether\footnote{Our ether has
nothing to do with the famous cryptocurrency of the same name.} and denote by~$\star$.

\item

Execution of a process takes place in a context, which maps names to process functions. A process
can invoke process functions by name, which allows for recursion. Replication of processes (provided
by the $!$-operator in the $\pi$- and the $\psi$-calculus) is not provided by the $\chi$-calculus,
since the availability of recursion makes it unnecessary.

\item

Processes can work with arbitrary data. The $\chi$-calculus uses higher-order abstract syntax (HOAS)
to make the full computational power of the host language available inside the process calculus. It
also uses HOAS for dealing with locally created channels. Through the consequent use of HOAS, all
naming issues are handled by the host language.

\end{itemize}

\input{session}

\end{document}
