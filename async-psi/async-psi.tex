\documentclass{article}
\usepackage[margin=2.5cm]{geometry}

\usepackage{kpfonts}
\usepackage{amsmath}
\usepackage{mathtools}
\usepackage{hyperref}
\usepackage{todonotes}
\usepackage{xspace}
\usepackage{listings}

\newcommand{\pOut}{{!}}
\newcommand{\bOut}{{!\!!}}
\newcommand{\pIn}{{?}}
\newcommand{\bIn}{{?\!?}}
\newcommand{\new}[2]{(\nu\!#1) #2}
\newcommand{\freshFor}{\mathrel{\#}}
\newcommand{\rec}{\mathtt{rec}}
\newcommand{\nil}{\mathbf{0}}
\newcommand{\invoke}[1]{\langle #1 \rangle}

\newcommand{\psicalculus}{psi-calculus\xspace}

\newcommand{\refinedBy}{\precsim}

\newtheorem{lemma}{Lemma}
\newtheorem{definition}{Definition}

\lstset{basicstyle=\ttfamily\small}
\lstdefinelanguage{Coq}{
  morekeywords={Require,Import,Open,Scope,Section,Context,Inductive,forall,Fixpoint,match,with,end,End,Arguments,return,Definition,Fact,Qed}
  }

\DeclareMathOperator{\bn}{bn}
\DeclareMathOperator{\n}{n}
\DeclareMathOperator{\len}{len}
\DeclareMathOperator{\chain}{chain}

\begin{document}

\title{Asynchronous Embedded Psi-Calculus}
\author{Edsko de Vries}
\maketitle

\section{Motivation}

The \psicalculus \cite{DBLP:journals/corr/abs-1101-3262} is a parametric process
calculus with support for broadcast communication \cite{Borgström2015} and
equipped with a formally verified metatheory. Its parameterization allows for
modelling of various cryptographic and other primitives, and its definition of
broadcast communication is very general. As such, it is a suitable candidate for
modelling the Proas algorithm and proving some of its properties. Its metatheory
includes a definition of bisimularity, along with a formal proof that
bisimularity is ``sound'', in the sense two bisimilar processes are
interchangeable in any context (sound with respect to a high-level observational
equivalence).

Unfortunately, such soundness comes at a cost. The relation between processes
(bisimilarity) must be detailed and is often complex; the definition of
bisimilarity for the \psicalculus in particular is non-trivial. Consequently,
proving bisimilarity between processes is often difficult and must be supported
by additional formal infrastructure. Although such infrastructure exists for the
\psicalculus in the form of up-to techniques, it currently does not support the
broadcast extension \cite{AmanPohjola:2016:BUT:2854065.2854080}.

Secondly, in order to be able to use \emph{any} kind of sound theory (one that
would allow us to conclude that we can replace a process by another in an
arbitrary context), we would have to fully characterize the behaviour of those
processes. This is a proof obligation which we currently don't have the
resources to take on, and instead we will only give partial specification.

On the positive side giving up on full specifications allows us to be more
flexible in what kind of calculus we use. When we use the \psicalculus directly,
every additional operation (such as encryption) that isn't explicitly supported
in the calculus must be encoded using the \psicalculus parameters. If however we
don't intend to use the existing \psicalculus metatheory, and don't intend to do
any metatheoretical reasoning ourselves (and instead only intend to reason about
concrete processes), we can define a process calculus that reuses infrastructure
from the host language (such as Isabelle) to give us additional power. For
example, in the Verdi project \cite{Wilcox:2015:VFI:2737924.2737958} processes
are simply host-language functions from inputs to outputs.

Embedding host-language functions in a calculus is known as higher order
abstract syntax (HOAS), and is typically associated with three problems in
formal verification. First, for a higher order abstract syntax the use of HOAS
is problematic as it requires a non-positive datatype. Since the psi-calculus is
first order, this is not an issue. Second, HOAS introduces ``exotic'' terms that
extend the power of the calculus; often these terms are undesirable but in our
case this is precisely the reason for introducing HOAS. We do have to be careful
that processes cannot distinguish one ``fresh'' name from another; it suffices
to require that the functions are equivariant. Finally, HOAS makes it difficult
or impossible to do induction over the syntax of the calculus (although there
are workarounds \cite{Röckl2001}). Since we don't intend to do metatheoretical
reasoning, but only plan to reason about concrete processes, I don't think that
this will be an issue.

Having said all that, it \emph{is} useful to stay as close to the \psicalculus
as possible, so that if in the future we do want to take advantage of the
\psicalculus metatheory the transition is not more difficult than it needs to
be. For this reason in this document we introduce an embedded process calculus
(``embedded'' in the sense that it is embedded in a host language with a
function space), which is closely modelled after the \psicalculus as it is
defined in \cite{Borgström2015}. The embedded nature of the calculus however
means that we can simplify the \psicalculus in various ways; this is documented
in detail in Section~\ref{sec:deviations}.

\section{Deviations from the \psicalculus}
\label{sec:deviations}

Since we want to work with a semantics in which we can model network
limitations, we work in an asynchronous process calculus (i.e., the output
process does not have a continuation). In an asynchronous process calculus there
is no guaranteed message delivery and no guaranteed message ordering. It
\emph{does} guarantee that messages don't get duplicated, but this is also
straight-forward to guard against in ``real'' code and moreover likely less
of a problem. The translation from an asynchronous calculus to the corresponding
synchronous calculus is trivial.

The \psicalculus as defined in
\cite{DBLP:journals/corr/abs-1101-3262,Borgström2015} uses the classical
``replication'' operator $!P$, modelling an unbounded number of instances of $P$
in parallel. This operator is convenient for modelling but less convenient for
interpretation. The \psicalculus workbench
\cite{Borgstrom:2015:PWG:2724585.2682570} uses named processes, effectively
introducing general recursion; we do the same. It is important that these named
processes are actually (host language) \emph{functions} from values to
processes, so that we can model things like process state.
Appendix~\ref{app:namedproc} shows one way in which we might model this formally
(the example uses Coq); I experimented also with using PHOAS for this purpose
but it turned out to be non-trivial (\emph{cf}.
\cite{Chlipala:2010:VCI:1706299.1706312}).

The rules for point-to-point communication in the \psicalculus rely on a
name-matching predicate $\xleftrightarrow{.}$; this is used for instance to
express that $\mathtt{fst}(c,d)$ and $c$ denote the same channel. Our use of
HOAS means that $\mathtt{fst}(c,d)$ will simply \emph{evaluate} to $c$ and hence
we don't need the name matching construct. Formally, we instantiate
$\xleftrightarrow{.}$ to be diagonal relation on the set of names (name
equality).

Similarly, the rules for broadcast connectivity use two predicates
$\overset{.}{\prec}$ and $\overset{.}{\succ}$, denoting ``output connectivity''
and ``input connectivity'' respectively. We simplify this to the trivial case:
we only use a single broadcast channel, that all processes can send to and
receive from. Formally, if $b$ is the single broadcast channel, we instantiate
both $\overset{.}{\prec}$ and $\overset{.}{\succ}$ to be trivial diagonal
relation on the set $\{b\}$.

In the \psicalculus the relations $\xleftrightarrow{.}$, $\overset{.}{\prec}$
and $\overset{.}{\succ}$ are all parameterized by the ``frame'' of the process;
a set of formulas that are asserted to be true at any point during the process'
lifecycle. Frames are also used to determine the branch to take in a case
statement. Since our instantiations of $\xleftrightarrow{.}$,
$\overset{.}{\prec}$ and $\overset{.}{\succ}$ are all trivial, and pattern
matching is taken care of by evaluation, we have no use for frames and omit them
from the calculus. Formally, we instantiate $\mathbf{A} = \{\top\}$, containing
only the trivial assertion, let $\mathbf{1} = \top$ and $\top \otimes \top =
\top$.

We follow the presentation of the psi-calculus from \cite{Borgström2015}
closely. In particular, we use nominal logic to deal with name binding
\cite{Pitts2001,DBLP:journals/corr/abs-0809-3960}. As stated, however, we use
HOAS for the input prefix, so that we can take advantage of the expressive power
of the host language to express processes. We do insist that these functions are
equivariant, to ensure that fresh names cannot be distinguished from each other.

The resulting calculus closely follows the \psicalculus, although the
resemblance is somewhat misleading: the introduction of HOAS significantly
extends the expressive power of the calculus. If and when we decide to use the
\psicalculus as stated in the literature, along with its metatheory, we need to
find ways to model all the primitives that we use in our processes.

\section{Formal Definition}

\subsection{Syntax}

\begin{tabular}{llll}
$P, Q$ & $::=$ && Process \\
  && $\nil$               & Nil                \\
  && $c \pOut V$          & Output             \\
  && $c \pIn \mathcal{F}$ & Input              \\
  && $\new{a}{P}$         & Restriction        \\
  && $P \parallel Q$      & Parallel           \\
  && $\bOut V$            & Broadcast output   \\
  && $\bIn \mathcal{F}$   & Broadcast input    \\
  && $\invoke{A} V$       & Process invocation \\
\\
\end{tabular}

We let $a, b, c$ range over channels, $V$ ranges over values, and $\mathcal{F}$
ranges over equivariant functions from values to processes, and $\alpha$ over
arbitrary actions. We allow arbitrary values in the host language.

\subsection{Operational Semantics}

Let $P, Q$ range over processes; $\mathcal{F}$ range over equivariant functions
from values to processes; $c, a, b$ over channels; $V$ over values.

\begin{itemize}

\item Point-to-point communication

\begin{equation*}
%
% PIn
%
\frac{
}{
c \pIn \mathcal{F} \xrightarrow{c \pIn V} \mathcal{F}(V)
}
\mathtt{In}
%
\qquad
%
% POut
%
\frac{
}{
c \pOut V. P \xrightarrow{c \pOut V} P
}
\mathtt{Out}
%
\qquad
%
\frac{
P \xrightarrow{\new{\bar{a}}{c \pOut V}} P' \qquad
Q \xrightarrow{c \pIn V} Q' \qquad
\bar{a} \freshFor Q
}{
P \parallel Q \xrightarrow{\tau} \new{\bar{a}}{(P' \parallel Q')}
}
\mathtt{Com}
%
\end{equation*}
%
\begin{equation*}
\frac{
P \xrightarrow{\new{\bar{a}}{c \pOut V}} P' \qquad
b \freshFor \bar{a}, c \qquad
b \in \n(V)
}{
\new{b}{P} \xrightarrow{\new{\bar{a},b}{c \pOut V}} P'
}
\mathtt{Open}
\end{equation*}
%
where
%
\begin{itemize}
\item Symmetric version of $\mathtt{Com}$ elided.
\item Rule $\mathtt{Com}$ doubles as a closing rule.
\end{itemize}

\todo[inline]{The psi calculus paper states some additional assumptions here,
but I \emph{think} they online pertain to frames, which we make no use of. But
we may need an additional freshness requirement somewhere.}

\item Broadcast communication

\begin{equation*}
%
% BOut
%
\frac{
}{
\bOut V. P \xrightarrow{\bOut V} P
}
\mathtt{BrOut}
%
\qquad
%
% BrIn
%
\frac{
}{
\bIn \mathcal{F} \xrightarrow{\bIn V} \mathcal{F}(V)
}
\mathtt{BrIn}
%
\qquad
%
% BCom
%
\frac{
P \xrightarrow{\new{\bar{a}}{\bOut V}} P' \qquad
Q \xrightarrow{\bIn V} Q' \qquad
\bar{a} \freshFor Q
}{
P \parallel Q \xrightarrow{\new{\bar{a}}{\bOut V}} P' \parallel Q'
}
\mathtt{BrCom}
\end{equation*}
%
\begin{equation*}
%
% BrOpen
%
\frac{
P \xrightarrow{\new{\bar{a}}{\bOut V}} P' \qquad
b \freshFor \bar{a} \qquad
b \in \n(V)
}{
\new{b}{P} \xrightarrow{\new{\bar{a},b}{\bOut V}} P'
}
\mathtt{BrOpen}
%
\qquad
%
% Close
%
\frac{
P \xrightarrow{\new{\bar{a}}{\bOut V}} P'
}{
P \xrightarrow{\tau} \new{\bar{a}}{P'}
}
\mathtt{BrClose}
\end{equation*}
%
\begin{equation*}
%
% BrMerge
%
\frac{
P \xrightarrow{\bIn V} P' \qquad
Q \xrightarrow{\bIn V} Q'
}{
P \parallel Q \xrightarrow{\bIn V} P' \parallel Q'
}
\mathtt{BrMerge}
\end{equation*}
%
where
%
\begin{itemize}
\item Symmetric version of rule $\mathtt{BrCom}$ is elided; note that the psi
calculus paper also elides rule $\mathtt{BrOpen}$ (we stated it explicitly).
\item Rule $\mathtt{BrMerge}$ would not be necessary if we had an explicit rule
\begin{math}
\displaystyle
\frac{
P \equiv P' \xrightarrow{\alpha} Q' \equiv Q
}{
P \xrightarrow{\alpha} Q'
}
\end{math}.
If we want such a rule to be \emph{derivable} instead (which makes the semantics
easier to use) rule $\mathtt{BrMerge}$ is needed.
\item Rule $\mathtt{BrClose}$ is slightly more permissive than its equivalent in
the psi calculus presentation, where it only applies at the point where the
channel is bound. I'm not entirely sure why that restriction is in place; it
is unusual (the closing rule for point-to-point communication has no similar
restriction).
\item Since we assume the existance of a single, global, broadcast channel,
restriction does not apply to broadcast channels. However, it \emph{is} possible
to extrude a point-to-point communication channel through a broadcast communication.
\item The rules for broadcast communication together mean that a single
broadcast output can be paired with any number (zero or more) inputs, and once
paired  (i.e., after the action) the output is no longer available. No
assumptions can be made about who will receive the message, and so broadcast
communication is ``unreliable'' in the sense that message may be ``lost''.
Moreover, since broadcast output, like regular output, is asynchronous,
broadcast communication is unordered.
\end{itemize}

\item Congruence rules

\begin{equation*}
%
% Par
%
\frac{
P \xrightarrow{\alpha} P' \qquad
\bn(\alpha) \freshFor Q
}{
P \parallel Q \xrightarrow{\alpha} P' \parallel Q
}
\mathtt{Par}
%
\qquad
%
% Scope
%
\frac{
P \xrightarrow{\alpha} P' \qquad
a \freshFor \alpha
}{
\new{a}{P} \xrightarrow{\alpha} \new{a}{P'}
}
\mathtt{Scope}
%
\qquad
\end{equation*}

Symmetric version of $\mathtt{Par}$ elided.

\end{itemize}

\section{Stating properties}

\todo[inline]{This section serves as a rough sketch only on how we might state
properties.}
\newcommand{\node}{\mathtt{Node}}

One of the main properties we want to state and prove is that a node can always
evolve to a state where its internal chain matches the chain of the environment
(given some compatibility conditions). If $\node$ is the function from a node's
internal state to the corresponding \psicalculus process, for the published
Praos algorithm we can state this simply as:
%
\begin{lemma}[Praos]
In any state $s$ and for any chain $c$ such that $\len(c) > \len(\chain(s))$,
there exists state $s'$ such that
\begin{equation*}
\node_P(s) \xrightarrow{\bIn c} \node_P(s')
\end{equation*}
and $\chain(s') = c$.
\end{lemma}
%
For the ``block Proas'' algorithm, the property is somewhat more complicated
since the node might be missing blocks.
%
\begin{lemma}[Block Praos]
\label{lem:blockpraos}
In any state $s$ and for any block $b$ such that $b$ is a valid successor of
$\chain(s)$ (though perhaps not the \emph{immediate} successor), there exists
$s', s'', \mu^*$ such that
\begin{equation*}
\node_B(s) \xrightarrow{\bIn b} \node(s') \xRightarrow{\mu^*} \node_B(s'')
\end{equation*}
and $b \in \chain(s'')$.
\end{lemma}
%
We can also try to relate the two implementations more directly. For example,
we could state that whenever the Praos node broadcasts a chain, the Block Praos
nodes will broadcast a series of blocks using the following coinductive
definition:
%
\begin{definition}[Simulation]
We say that $\node_B(s_B)$ can simulate $\node_P(s_P)$, written
$\node_P(s_P) \refinedBy \node_B(s_B)$, iff for all $s_P', c$ such that
\begin{equation*}
\node_P(s_P) \xrightarrow{\bOut c} \node_P(s_P')
\end{equation*}
there exists $s_B'$ and blocks $b^*$, a suffix of $c$, such that
\begin{equation*}
\node_B(s_B) \xRightarrow{\bOut b^*} \node_B(s_B')
\end{equation*}
and $\node_P(s_P') \refinedBy \node_B(s_B')$.
\end{definition}
%
We would then prove that
%
\begin{lemma}[Block Praos simulates Praos]
\label{lem:BlockSimPraos}
For all $s_P, s_C$ such that $\chain(s_P) = \chain(s_B)$,
\begin{equation*}
\node_P(s_P) \refinedBy \node_B(s_B)
\end{equation*}
\end{lemma}
%
We may of course also wish to prove something about how the other actions
relate, or prove something in the other direction too (bisimulation rather than
simulation). Note that such a (bi)simulation doesn't necessarily obsolete the
need for additional lemmas such as Lemma~\ref{lem:blockpraos}, since in the
proof of Lemma~\ref{lem:BlockSimPraos} we can pick paths in which no messages
get lost.

\appendix

\section{Appendix: Named Processes}
\label{app:namedproc}

Example Coq development of defining a process calculus with named
process invocation. Relies on some supporting definitions for vectors and
dependent vectors, shown in Appendix~\ref{app:coqaux}.

\begin{lstlisting}[language=Coq]
Section Syntax.

  Context {n   : nat}.
  Context {typ : Vec Set n}.

  Inductive Proc : Type :=
      Done : Proc
    | Out  : nat -> Proc -> Proc
    | Inv  : forall (i : Fin n), nth i typ -> Proc.

  Definition Env := DepVec (fun a => a -> Proc) typ.

End Syntax.

Arguments Proc {n} typ.
Arguments Env  {n} typ.
Arguments Inv  {n} typ.

Section Semantics.

  Context {n   : nat}.
  Context {typ : Vec Set n}.
  Context {env : Env typ}.

  Inductive HasTrace : Proc typ -> list nat -> Prop :=
      TNil  : forall P, HasTrace P nil
    | TOut  : forall n P t, HasTrace P t ->  HasTrace (Out n P) (cons n t)
    | TInv  : forall i x t, HasTrace (dnth i env x) t -> HasTrace (Inv _ i x) t.

End Semantics.

Arguments HasTrace {n} {typ} env.

Section Example.

  Definition typ : Vec Set 2 := (nat, (nat, tt)).

  Definition left  (n : nat) : Proc typ := Out n (Inv typ (FS FZ) (n + 1)).
  Definition right (n : nat) : Proc typ := Out n (Inv typ     FZ  (n + 1)).

  Definition env : Env typ := (left, (right, tt)).

  Fact trace123 : HasTrace env (left 1) (1 :: 2 :: 3 :: nil).
    repeat constructor.
  Qed.

End Example.
\end{lstlisting}

\section{Appendix: Supporting definitions}
\label{app:coqaux}

Some supporting Coq infrastructure for the development in Appendix~\ref{app:namedproc}.

\begin{lstlisting}[language=coq]
Section Vec.

  Context (A : Type).

  Inductive Fin : nat -> Set :=
      FZ : forall {n}, Fin (S n)
    | FS : forall {n}, Fin n -> Fin (S n).

  Fixpoint Vec (n : nat) : Type :=
    match n with
      O    => unit
    | S n' => A * Vec n'
    end.

  Fixpoint nth {n : nat} (i : Fin n) : Vec n -> A :=
    match i with
      FZ    => fun xs => match xs with (x, xs') => x          end
    | FS i' => fun xs => match xs with (x, xs') => nth i' xs' end
    end.

End Vec.
Arguments nth {A} {n} i.

Section DepVec.

  Fixpoint DepVec (f : Set -> Set) {n : nat} : Vec Set n -> Set :=
    match n return Vec Set n -> Set with
      0    => fun _  => unit
    | S n' => fun As => match As with (A, As') => (f A * DepVec f As')%type end
    end.

  Fixpoint dnth {n : nat} {f : Set -> Set} (i : Fin n) :
     forall {xs : Vec Set n}, DepVec f xs -> f (nth i xs) :=
    match i in Fin n return forall (xs : Vec Set n), DepVec f xs -> f (nth i xs) with
      FZ    => fun As =>
                 match As with (A, As') => fun xs =>
                   match xs with (x, xs') => x end
                 end
    | FS i' => fun As =>
                 match As with (A, As') => fun xs =>
                   match xs with (x, xs') => dnth i' xs' end
                 end
    end.

End DepVec.
\end{lstlisting}

\bibliographystyle{acm}
\bibliography{references}

\end{document}
